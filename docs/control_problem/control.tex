\documentclass{article}
\usepackage[margin=20mm]{geometry}
\usepackage{setspace}
\bibliographystyle{unsrt}

\title{Notes on the BISICLES control problem}
\author{Stephen L Cornford}
\date{Aug 20 2012}
\begin{document}
\maketitle
\onehalfspacing
BISICLES requires two fields, the rate factor $A$ and the sliding
coefficient $C$, in addition to the ice thickness $H$ and the
bedrock topography $r$ in order to solve the stress balance equation.
Neither of these are directly observed, but we do typically have 
observations of either the ice velocity $\vec{u}_{\rm obs}$, or speed, and of thinning rate. 
We adopt a  control method similar to those reported elsewhere
\cite{JTBB09:doi:10.3189/002214309788608705,JoSH10:doi:10.1029/2010GL044819,Maca93,MRSL10:doi:10.1029/2010GL043853},
that is, a gradient based optimization method which makes use of the model adjoint equations.
We don't seek $A$ itself, but rather take $A$ to be a function of an imposed temperature field $T$ and seek
a multiplier, $\phi$ of the effective viscosity (which contains a factor $A^{1/n}(T)$). The optmization
problem we solve is inevitably ill-posed, so we employ Tikhnonov regularization, introducing
a bias toward smoothly varying solutions.
 
For simplicity, the method will be described for a 1D ice sheet, and the 2D equivalents given in the
appendix.

\section{Model equations}


We have a stress balance equation
\begin{equation}
\label{eq::sb}
\frac{\partial}{\partial x} \left ( \phi H \bar{\mu}  \frac{\partial u}{\partial x} \right ) - Cu = \rho g H \frac{\partial s}{\partial x} 
\end{equation}
and a mass transport equation
\begin{equation}
\frac{\partial H }{\partial t} + \frac{\partial}{\partial x} \left (\bar{u} H \right ) - M = 0
\end{equation}
plus boundary conditions. Here, $H \bar{\mu}$ is the vertically integrated effective viscosity, which is computed from $A$ and
$u$ through Glen's flow low. $u$ is the velocity 
at the base of the ice, and $\bar{u}$ is the verticall averaged velocity. We will assume that $u \approx \bar{u}$ for now. $M$
is the total mass source (ie surface accumulation and ablation plus sub-shelf melting etc).

\section{Objective function and gradient without regularization}

We first consider the $C$ and $\phi$ which minimise the mismatch between the model speed and observed speed.
In that case, our objective function $J$ would be, absent any kind of regularization, 
\begin{equation}
J_{\rm 1} = \int _{x_0} ^{x_1} \frac{1}{2\sigma^2}(|u| - |u_{\rm obs}|)^2 \,{\rm d}x .
\end{equation}
$\sigma^2$ is the variance in the error of $|u_{\rm obs}|$, which we assume has spatially uncorrelated Gaussian statistics.

To make use of a gradient-based optimization method, we need to compute
the functional (Gateaux) derivatives $\frac{\delta J}{\delta C}$, and $\frac{\delta J}{\delta \phi}$ so to that end we add a term to $J$,
\begin{equation}
J_{\rm 2} =  \int _{x_0} ^{x_1} \lambda \left \{ \frac{\partial}{\partial x} \left ( \phi H \bar{\mu}  \frac{\partial u}{\partial x} \right ) - Cu - \rho g H \frac{\partial s}{\partial x} \right\} \,{\rm d}x .
\end{equation}
where $\lambda$ is an undetermined Lagrange multiplier. We then seek solutions to 
\begin{equation}
\label{eq::gradJaug}
\left ( \begin{array}{l}
\frac{\delta J}{\delta \phi} \\
\frac{\delta J}{\delta C} \\
\frac{\delta J}{\delta u } \\
\frac{\delta J}{\delta \lambda} \\
\end{array} \right ) = 0.
\end{equation}

The final row of \ref{eq::gradJaug} is satisfied when provided that $u$ is a solution to the stress balance equation.
The left hand sides of the first two rows of (\ref{eq::gradJaug}) can be written:
\begin{equation}
\label{eq::gradJ}
\left (
\begin{array}{l}
\frac{\delta J}{\delta \phi} \\
\frac{\delta J}{\delta C} 
\end{array} 
\right ) = \left ( 
\begin{array}{l}
  - \bar{\mu} H \frac{\partial u}{\partial x}\frac{\partial \lambda}{\partial x} \\
  - \lambda u
\end{array} 
\right )
\end{equation}
where $\lambda$ is the solution to the last row. As for that third row, if we neglect the dependence of $\mu$ on $u$,
we have the \emph{adjoint equation} 
\begin{equation}
\label{eq::adj}
\frac{\partial}{\partial x} \left ( \phi H \bar{\mu}(u)  \frac{\partial \lambda}{\partial x} \right ) - C \lambda  = \frac{1}{\sigma^2} \left ( \frac{|u_{obs}|}{u} - 1 \right ) u 
\end{equation}
plus its boundary conditions
\begin{equation}
\lambda(x_0) = \lambda(x_1) = 0
\end{equation}
Equation (\ref{eq::adj}) is linear in $\lambda$ and 
has the happy property of having the same left hand side as the stress-balance equation (\ref{eq::sb}), but with $\lambda$
swapped for $u$. In other words, the
stress balance equation is self-adjoint if (and only if) we neglect the dependence of $\mu$ on $u$, and we will be
able to use the same methods to solve (\ref{eq::adj}) as we use to solve (\ref{eq::sb})

We are now in a position to use a gradient-based optimization method to mimimise $J$. BISICLES uses a (nonlinear) conjugate gradient method, 
but other methods ought to work too. Whichever one we choose, we will need to compute the functional derivatives of
$J$, which we can do for a given $C$ and $\phi$ as follows:
\begin{enumerate}
\item{Solve (\ref{eq::sb}) for $u$ given $C$ and $\phi$ in order to compute $\bar{\mu}$.}
\item{Solve the adjoint equation (\ref{eq::adj}) for $\lambda$ given $C$,$\phi$ and $\bar{\mu}$.}
\item{Compute the functional derivatives using (\ref{eq::gradJ}).}
\end{enumerate}

\section{Objective function and gradient with regularization}

The optimization problem of the previous section is ill-posed in at least two senses.
First, we are seeking two scalar fields ($C$ and $\phi$) given one scalar field 
of data ($|u _{obs}|$). Second, even if we were just seeking $C$, say, we might expect the
problem to be ill-posed - imagine for example that we added a single, tiny sticky spot to
$C$, then we would expect that to make little difference to the field $u$\footnote{SLC: do some meaningful analysis here, it shouldn't be 
difficult to show that $u(C)$ is compact.}

To ameliorate this ill-posedness, we can add some penalty functions to the objective function. To
bias in favour of smooth $C$ and $\phi$, we can add 
\begin{equation}
J_{3}  = \alpha _C ^2 \left [ \int _{x_0} ^{x_1} \left ( \frac{\partial C}{ \partial x} \right )^ 2 dx + (C(x_0) - C_0) +(C(x_1) - C_0)   \right ] 
\end{equation} 
\begin{equation}
J_{4}  = \alpha _\phi ^2 \left [ \int _{x_0} ^{x_1} \left ( \frac{\partial \phi}{ \partial x} \right )^ 2 dx + (\phi(x_0) - \phi_0) +(\phi(x_1) - \phi_0)   \right ] 
\end{equation} 

$C_0$,$C_1$,$\phi_0$ and $\phi _1$ are boundary data. BISICLES boundaries tend to be far from the action, at
ice divides and in the ocean, so we set  $\phi_0 = \phi _1 = 1$ and $C_0,C_1$ to be some large value (on land),
or zero (for floating ice). The coefficients $\alpha _C ^2$ and $\alpha _\phi ^2$ could be determined by some systematic means 
(e.g. cross validation, finding a critical point in the $L-$curve) but they can also be regarded
as a characteristic length scales of variation in $C$ and $\phi$, and imposed on that basis. The larger they
are, the worse the fit between model and data will be.\footnote{We probably should support some systematic means of choosing
these parameters, but we also make use of the iterative regularization provided by CG, so needs a bit of thought.}

Adding $J_3$ and $J_4$ to the objective function, (\ref{eq::gradJ}) becomes:
\begin{equation}
\label{eq::gradJTikh}
\left (
\begin{array}{l}
\frac{\delta J}{\delta \phi} \\
\frac{\delta J}{\delta C} 
\end{array} 
\right ) = \left ( 
\begin{array}{l}
  - \bar{\mu} H \frac{\partial u}{\partial x}\frac{\partial \lambda}{\partial x^2} - \alpha^2_\phi \frac{\partial^2 \phi}{\partial x^2} \\
  - \lambda u - \alpha^2_C \frac{\partial^2 C}{\partial^2 x}
\end{array} 
\right ).
\end{equation}
In effect, we are solving a pair of Poisson equations in $\phi$ and $C$ with Robin boundary conditions.


\section*{Appendix}


\bibliography{control}


\end{document}
